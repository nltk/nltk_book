\documentclass[12pt]{article}
\usepackage{fullpage}

\def\nogap{\setlength\itemsep{.0in}\setlength{\parskip}{0in}}

\pagestyle{empty}
\begin{document}
\title{NLTK: A Natural Language Toolkit}
\author{Steven Bird \& Edward Loper}

% clunch:
%\date{Thursday December 12, 2001}

% lsa:
\date{Sunday January 6, 2002}

\maketitle

\section*{Overview}
\thispagestyle{empty}

The Natural Language Toolkit is a Python package that simplifies the
construction of programs that process natural language.  In
particular:

% Note: can I say NLP, or do I have to say natural language
% processing? 
\begin{itemize}\nogap

  \item It provides basic tools for manipulating data and performing
  tasks related to NLP.

  \item It defines standard interfaces between the different components
  of an NLP system.

  \item It provides an infrastructure for building new NLP systems.

\end{itemize}

 \noindent The toolkit's primary aim is to serve as a pedagogical
 tool; but it is also useful as a framework for implementing
 research-related programs.

% Say something about what platforms are supported..
\vspace{3mm}\noindent NLTK runs on most platforms, including Windows, OS X, Linux, and
UNIX.

\section*{Toolkit Contents}
\thispagestyle{empty}

\begin{itemize}\nogap
    \item \textbf{Python Modules} implement the basic data types,
      tools, and interfaces that make up the toolkit.

    \item \textbf{Tutorials} teach students how to use the toolkit, in the
      context of performing specific tasks.

    \item \textbf{Exercises and Problem Sets} help students learn more about
      various aspects of natural language processing.

    \item \textbf{Reference Documentation} provides precise definitions of the
      behavior of each module, interface, class, method, function, and
      variable defined by the toolkit.  

    \item \textbf{Technical Documentation} explains and justifies the
      toolkit's design and implementation. 
\end{itemize}

\section*{Contributing}
\thispagestyle{empty}

NLTK is an open source project, and we welcome any contributions.  We
deliberately structured NLTK to facilitate parallel development.  If
you are interested in contributing to NLTK, or have any ideas for
improvements, please talk to us, or send us email at
\texttt{edloper@gradient.cis.upenn.edu} and
\texttt{sb@unagi.cis.upenn.edu}.  

\section*{Modules}
\thispagestyle{empty}

NLTK currently includes the following modules.

{\small
\begin{itemize}\nogap

  \item Basics
    \begin{itemize}\nogap
      \item \texttt{token}: Basic classes for processing individual
            elements of text, such as words or sentences. 
      \item \texttt{tree}: Classes for representing tree structures
            over text (such as syntax trees and morphological trees).
      \item \texttt{probability}: Classes that encode frequency
            distributions and probability distributions.
    \end{itemize}

  \item Tagging
    \begin{itemize}\nogap
      \item \texttt{tagger}: A standard interface to tag each token of
            a text with supplementary information, such as its part of
            speech; and several implementations of that interface.
    \end{itemize}

  \item Parsing
    \begin{itemize}\nogap
      \item \texttt{parser}: A standard interface to produce trees
            representing the structure of texts.
      \item \texttt{chartparser}: A flexible parser implementation
            that uses a \emph{chart} to record hypotheses about
            syntactic constituents.
      \item \texttt{srparser\_template}: A partial implementation of a
            shift-reduce parser; completing the implementation was a
            student exercise.
      \item \texttt{chunkparser}: A standard interface for robust
            parsers used to identify non-overlapping linguistic groups
            (such as noun phrases) in unrestricted text.
      \item \texttt{rechunkparser}: A regular-expression based
            implementation of the chunk parser interface.
    \end{itemize}

  \item Text Classification
    \begin{itemize}\nogap
      \item \texttt{classifier}: A standard interface for classifying
          texts into categories.
      \item \texttt{classifier.feature}: A standard way of encoding
          the information used to make classification decisions.
      \item \texttt{classifier.naivebayes}: A text classifier
          implementation based on the Naive Bayes assumption.
      \item \texttt{classifier.maxent}: An implementation of the
          maximum entropy model for text classification; and
          implementations of the GIS and IIS algorithms for training
          the classifier.
      \item \texttt{classifier.featureselection}: A standard interface
          for choosing which features are relevant for making
          classification decisions.
    \end{itemize}

  \item Visualization
    \begin{itemize}\nogap 
      \item \texttt{draw.tree}: A graphical representation for tree
            structures, such as syntax trees and morphological trees. 
      \item \texttt{draw.tree\_edit}: A graphical interface used to
            build and modify tree structures.
      \item \texttt{draw.plot\_graph}: A graphical tool to graph
            arbitrary functions.
      \item \texttt{draw.chart}: An interactive graphical tool used to
            experiment with chart parsers.
    \end{itemize}

\end{itemize}
}
\section*{URL}

For more information about NLTK, or to download a copy, please
visit our web page: 

\texttt{http://nltk.sourceforge.net}

\end{document}

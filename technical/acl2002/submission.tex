\documentclass[11pt]{article}
\usepackage{acl02sub,url}
\title{NLTK: The Natural Language Tool Kit}
\author{
\begin{tabular}[t]{c}
Edward Loper\\
Department of Computer and Information Science\\
University of Pennsylvania\\
Philadelphia, PA 19104-6389, USA
\end{tabular}
\hspace{6pt}
\begin{tabular}[t]{c}
Steven Bird\\
Linguistic Data Consortium\\
University of Pennsylvania\\
Philadelphia, PA 19104-2608, USA
\end{tabular}
}
\summary{%
NLTK, the Natural Language Tool Kit, is a suite of program modules,
tutorials and problem sets.  NLTK covers symbolic and statistical
NLP, and is interfaced to annotated corpora.
Students augment and replace existing components, learning
structured programming by example, and manipulating sophisticated
models from the outset.  Along with extensive documentation and
problem sets, NLTK provides self-contained, ready-to-use CL
courseware.}
\paperid{Pxxxx}
\keywords{courseware, Python, corpora, keyword4?, keyword5?}
\contact{Edward Loper}
\conference{this paper has not been submitted to any other conferences}
\date{\today}

\begin{document}
\makeidpage
\maketitle

\begin{abstract}
Students in computational linguistics courses must often learn
a new programming language.  Where courses are offered in linguistics
departments, the students may be learning to program for the first time.
Many low-level ``housekeeping'' tasks must
be accomplished in order to do interesting projects.  At the same
time, teachers of computational linguistics courses sometimes feel
that they spend too much time teaching students to program, and not
enough time teaching the subject itself.  They may even avoid
programming assignments altogether.  However, we believe that it is
crucial for a first computational linguistics course to include a
strong practical component, in which students develop real programs to
solve real problems with real data.

Python is a new object-oriented scripting language which runs on all
platforms.  Python has been praised as "executable pseudocode", since
programs are so easy to write.  Recently, we have been developing
NLTK, an open-source Natural Language Toolkit written in Python.
In this presentation, we will motivate, describe and demonstrate NLTK.

NLTK, the Natural Language Tool Kit, is a suite of program modules,
tutorials and problem sets.  NLTK covers symbolic and statistical
natural language processing, and is interfaced to annotated corpora.
Students augment and replace existing NLTK components, learning
structured programming by example, and manipulating sophisticated
models from the outset.  Along with extensive documentation and
problem sets, NLTK provides self-contained, ready-to-use CL
courseware.
\end{abstract}

\section{Introduction}

Teachers of introductory courses on computational linguistics are
often faced with the challenge of setting up a practical programming
component for student assignments and projects.  This is difficult not
just because of the variety of data structures, but also because of
the diverse range of topics which may need to be included in the
syllabus.  A widespread practice is to teach multiple programming
languages, where each provides data structures and native functions
that are a good fit for the task at hand (e.g. Prolog for parsing,
Perl for corpus processing, a finite-state toolkit for morphological
analysis).  By relying on the built-in features of various languages,
the teacher avoids having to develop a lot of software infrastructure.
An unfortunate consequence is that a significant part of the course
must be devoted to teaching these languages.  Further, many
interesting projects require the languages to be bridged, e.g. a
project that involved syntactic parsing of corpus data from a
morphologically rich language would involve all three languages
mentioned above (Perl for file I/O, format conversions and overall
program control, with calls out to a finite state toolkit for
morphological analysis and to a Prolog engine for parsing).  It is
clear that these considerable overheads and shortcomings warrant a
fresh approach.

Apart from the practical component, computational linguistics courses
may depend on appropriate software in another way, namely for in-class
demonstrations.  This context calls for highly interactive graphical
user interfaces making it possible to view program state (e.g. the
chart of a chart parser), observe program execution step-by-step
(e.g. execution of a finite-state machine), and even make minor
modifications to programs in response to ``what if'' questions from
the class.  Because of these difficulties it is common to avoid live
demonstrations, and keep classes for theoretical presentations only.
Apart from being dull, this approach leaves students to solve
important practical problems on their own, or to deal with them less
efficiently in office hours.

In this paper we describe a new approach to the above challenges,
a streamlined and flexible way of organizing the practical component
of an introductory computational linguistics course.  We describe
NLTK, the Natural Language Tool Kit, which we have developed in
conjunction with a course we have taught at the University of Pennsylvania.

OVERVIEW OF THE PAPER

All materials discussed here are available under an open
source license from \url{nltk.sf.net}.

\section{Choice of programming language}

Requirements on a Programming Language

shallow learning curve
new programmers must get immediate rewards
support for rapid prototyping
we want to avoid the compilation step
self-documenting code
programs must be immediately comprehensible
support for good programming style
it must be easy to write well-structured programs
graphical user interface
the language must have a good, easy-to-use GUI

Python: object-oriented scripting

shallow learning curve
Python was designed to be easily learnt by children
support for rapid prototyping
Python is interpreted, with no compilation step
self-documenting code
Python has been called "executable pseudocode"
support for good programming style
Python is object-oriented (but not punitively so)
graphical user interface
Python has an interface to the tk GUI toolkit

\section{NLTK design principles}


\section{Structure of the toolkit}

Code, documentation, ...

\section{Basic modules}

\subsection{Token}

\subsection{Tree}

\subsection{Tagger}

\subsection{Probability}

\section{Advanced modules}

\subsection{ChunkParser}

\subsection{ChartParser}

\subsection{FSA}

\subsection{Classifier}


\section{Evaluation}

How we used NLTK at Penn (including the competition).

Strengths and weaknesses

\section{Other approaches}

Java: \cite{Hammond02}
-- no obligation to cite this as it is unpublished
(and I haven't seen a copy).

\section{Conclusion}

Overview of the paper.

Upbeat conclusion.

Invitation to participate.

\bibliographystyle{acl}
\bibliography{submission}

\end{document}
